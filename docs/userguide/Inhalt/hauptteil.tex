\chapter{Prozesse}
\label{cha:prozesse}
In diesem Kapitel werden ausgewählte Standartprozesse zur Übersicht und späteren Orientierung der Benutzer definiert.

\section{Benutzer registrieren}
\label{sec:process_register}
Jeder Helfer benötigt einen eigenen Benutzeraccount. Dieser kann auf der Startseite unter \glqq Registrieren\grqq ~angelegt werden. Der Benutzer muss zunächst durch einen Administrator freigeschaltet werden. Der Benutzer wird über E-Mail von der Freischaltung informiert. Nach dem ein Benutzer Freigeschaltet wurde wird der Administrator die entsprechenden Qualifikationen zuweisen. Wenn nicht alle bzw. die entsprechenden Qualifikationen zugewiesen wurden, sollten die Administratoren darauf aufmerksam gemacht werden.

\noindent \textit{Es ist zu empfehlen die selbe E-Mail Adresse wie bei Facebook zu verwenden. Hierdurch kann später das \glqq oneclick\grqq ~Login verfahren verwendet werden.}

\section{Dienst melden}
\label{sec:process_position_apply}
Für das Melden zu einem Dienst muss lediglich der entsprechende Tag gesucht werden. Bei Positionen mit entsprechender Voraussetzung an Qualifikationen kann der Benutzer sich direkt melden.

\noindent \textit{Ein Benutzer sollte sich bei einem Dienst zu mehreren Positionen melden. Die Administratoren werden nur einen bestätigen und haben somit mehr Flexibilität beim Besetzen der Dienste.}

\section{Dienst Aufteilen}
\label{sec:process_service_split}
Es ist möglich einzelne Dienste bzw. dessen Positionen in zwei Schichten aufzuteilen. Bereits geteilte Dienste sind an einem Kommentar der Position zu erkennen. Ist ein Dienst der geleistet werden will nicht aufteilt, werden die Administratoren durch einem freundlichen Hinweis per E-Mail dies einrichten.

\section{Dienst Absagen}
\label{sec:process_service_cancel}
Sollte ein Benutzer einen bereits gemeldeten Dienst absagen müssen, kann er dies, wenn noch nicht bestätigt, über \glqq Meldung zurückziehen\grqq ~(siehe Kapitel \ref{cha:dienste} \nameref{cha:dienste}) selber durchführen. Ist der Dienst bereits bestätigt, muss mit den Administratoren Kontakt aufgenommen werden.